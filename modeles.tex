\begin{block}{Modèles pour logiciels auteurs}
\textbf{Standard} : modèle-vue-contrôleur, modèle-vue-présenteur, document-présentation-instrument, modèle-vue-modèle de vue, présentation-abstraction-contrôle, programmation fonctionnelle réactive.
Servent à la \textbf{séparation des responsabilités} lors du développement et principalement à la manière dont une interaction utilisateur affecte le modèle de données et du retour sur affichage.~\\

Problématique de l'\textbf{édition temps-réel avec contraintes}~\cite{laffra2012object}.~\\

Dans jeux-vidéo, modèle courant : entité-composant-système.
\begin{itemize}
\item \textbf{Entité} : objet ou identifiant trivial, auquel on associe des composants. 
\item \textbf{Composant} : contient des données.
\item Une famille commune de composants (rendu, physique, son) est gérée 
par un même \textbf{Système}.
\end{itemize}

\end{block}