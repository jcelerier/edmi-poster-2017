\setbeamerfont*{block title}{size=\Large,series=\bfseries}
\begin{columns}[t]
   \begin{column}{\sepwid}\end{column}
     \begin{column}{\onecolwid}
      \begin{block}{Problématique}
          \begin{columns}[t]
              \begin{column}{\onecolwid}\justify
                  On cherche à introduire une sémantique permettant l'exécution de sous-parties de partitions interactives sur des machines distinctes, de manière à permettre de nouvelles formes d'écriture musicale.
                \end{column}
            \end{columns}        
      \end{block}
     \end{column}
     \begin{column}{\sepwid}\end{column}
     \begin{column}{\twocolwid}
         \begin{block}{Méthode}             
             \begin{columns}[t]	                 
                 \begin{column}{\onecolwid}\justify
                     On abstrait les hôtes réseau comme clients dans des groupes que manipulent les compositeurs. Des annotations de répartition sont introduites pour les éléments de synchronisation ainsi que pour les structures temporelles.
                     \end{column}
                     \begin{column}{\onecolwid}\justify
                     Sont notamment définis des modes synchrones et asynchrones que peut choisir le compositeur de manière à privilégier la latence, ou bien la synchronisation entre hôtes, en fonction des contraintes de sa pièce.
                        \end{column}
                \end{columns}                 
            \end{block}
      \end{column}
      \begin{column}{\sepwid}\end{column}
      \begin{column}{\onecolwid}
          \begin{block}{Résultats}
          	\begin{columns}[t]
          		\begin{column}{\onecolwid}\justify
          			La modélisation de la méthode de répartition se fait elle-même dans le paradigme des partitions interactives.
          			Une implémentation préliminaire, en architecture client-serveur, est offerte en tant qu'extension au logiciel.
          		\end{column}
          	\end{columns}        
            \end{block}
        \end{column}         
   \begin{column}{\sepwid}\end{column}		
\end{columns}

\setbeamerfont*{block title}{size=\Large}