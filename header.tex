\setbeamerfont*{block title}{size=\Large,series=\bfseries}
\begin{columns}[t]
   \begin{column}{\sepwid}\end{column}
     \begin{column}{\onecolwid}
      \begin{block}{Problématique}
          \begin{columns}[t]
              \begin{column}{\onecolwid}\justify
                  Étant donné une partition interactive, quelles sémantiques peut-on introduire pour permettre l'exécution de sous-parties de cette partition sur des machines distinctes, en permettant de nouvelles formes d'écriture.
                \end{column}
            \end{columns}        
      \end{block}
     \end{column}
     \begin{column}{\sepwid}\end{column}
     \begin{column}{\twocolwid}
         \begin{block}{Méthode}             
             \begin{columns}[t]	                 
                 \begin{column}{\onecolwid}\justify
                     Abstraction des hôtes réseau sous forme de clients dans des groupes que manipulent les compositeurs. On introduit des annotations de répartition pour les éléments de synchronisation ainsi que pour les structures temporelles.
                     \end{column}
                     \begin{column}{\onecolwid}\justify
                     	On définit notamment des modes synchrones et asynchrones que peut choisir le compositeur de manière à privilégier la latence, ou bien la synchronisation entre hôtes.
                        \end{column}
                \end{columns}                 
            \end{block}
      \end{column}
      \begin{column}{\sepwid}\end{column}
      \begin{column}{\onecolwid}
          \begin{block}{Résultats}
          	\begin{columns}[t]
          		\begin{column}{\onecolwid}\justify
          			Une implémentation préliminaire, en architecture client-serveur, est offerte en tant qu'extension au logiciel i-score.
          		\end{column}
          	\end{columns}        
            \end{block}
        \end{column}         
   \begin{column}{\sepwid}\end{column}		
\end{columns}

\setbeamerfont*{block title}{size=\Large}