%%%%%%%%%%%%%%%%%%%%%%%%%%%%%%%%%%%%%%
% This poster uses a theme taken from Nathaniel Johnston
% http://www.nathanieljohnston.com/2009/08/latex-poster-template/
%%%%%%%%%%%%%%%%%%%%%%%%%%%%%%%%%%%%%%

\documentclass[final]{beamer}
\usepackage[scale=1.12]{beamerposter}
\usepackage{graphicx}
\usepackage{pstricks}
\usepackage{tikz}
\usepackage{polyglossia}
\setdefaultlanguage{french} 
\usepackage{diagbox}
\usepackage{xspace}
\usepackage{microtype}
\usepackage{default}
\usepackage{listings}
\usepackage{booktabs}
\usepackage[backend=biber]{biblatex}
\addbibresource{document.bib}
\renewcommand*{\bibfont}{\footnotesize}

\newcolumntype{C}[1]{>{\centering\arraybackslash}m{#1}}   %% centered
\newcolumntype{R}[1]{>{\raggedright\arraybackslash}m{#1}}  %% right aligned
\newcolumntype{L}[1]{>{\raggedleft\arraybackslash}m{#1}}  %% right aligned


%-----------------------------------------------------------
% Define the column width and poster size
% To set effective sepwid, onecolwid and twocolwid values, first choose how many columns you want and how much separation you want between columns
% The separation I chose is 0.024 and I want 4 columns
% Then set onecolwid to be (1-(4+1)*0.024)/4 = 0.22
% Set twocolwid to be 2*onecolwid + sepwid = 0.464
%-----------------------------------------------------------

\newlength{\sepwid}
\newlength{\onecolwid}
\newlength{\twocolwid}
\setlength{\paperwidth}{48in}
\setlength{\paperheight}{36in}
\setlength{\sepwid}{0.024\paperwidth}
\setlength{\onecolwid}{0.22\paperwidth}
\setlength{\twocolwid}{0.464\paperwidth}
\setlength{\topmargin}{-0.5in}
\usetheme{confposter}
\usepackage{exscale}

%-----------------------------------------------------------
% The next part fixes a problem with figure numbering. Thanks Nishan!
% When including a figure in your poster, be sure that the commands are typed in the following order:
% \begin{figure}
% \includegraphics[...]{...}
% \caption{...}
% \end{figure}
% That is, put the \caption after the \includegraphics
%-----------------------------------------------------------

\usecaptiontemplate{
\small
\structure{\insertcaptionname~\insertcaptionnumber:}
\insertcaption}

%-----------------------------------------------------------
% Define colours (see beamerthemeconfposter.sty to change these colour definitions)
%-----------------------------------------------------------

\setbeamercolor{block title}{fg=DarkGray,bg=white}
\setbeamercolor{block body}{fg=DarkGray,bg=white}
\setbeamercolor{block alerted title}{fg=DarkGray,bg=LightGray}
\setbeamercolor{block alerted body}{fg=DarkGray,bg=LightGray}
%-----------------------------------------------------------
% Name and authors of poster/paper/research
%-----------------------------------------------------------

\title{Exécution répartie de partitions interactives}
\author{Jean-Michaël Celerier}
\institute{Laboratoire Bordelais de Recherche en Informatique, Blue Yeti}

\begin{document}
\begin{frame}[fragile,t]    
   \setbeamerfont*{block title}{size=\Large,series=\bfseries}
\begin{columns}[t]
   \begin{column}{\sepwid}\end{column}
     \begin{column}{\onecolwid}
      \begin{block}{Problématique}
          \begin{columns}[t]
              \begin{column}{\onecolwid}\justify
                  Étant donné une partition interactive, quelles sémantiques peut-on introduire pour permettre l'exécution de sous-parties de cette partition sur des machines distinctes, en permettant de nouvelles formes d'écriture.
                \end{column}
            \end{columns}        
      \end{block}
     \end{column}
     \begin{column}{\sepwid}\end{column}
     \begin{column}{\twocolwid}
         \begin{block}{Méthode}             
             \begin{columns}[t]	                 
                 \begin{column}{\onecolwid}\justify
                     Abstraction des hôtes réseau sous forme de clients dans des groupes que manipulent les compositeurs. On introduit des annotations de répartition pour les éléments de synchronisation ainsi que pour les structures temporelles.
                     \end{column}
                     \begin{column}{\onecolwid}\justify
                     	On définit notamment des modes synchrones et asynchrones que peut choisir le compositeur de manière à privilégier la latence, ou bien la synchronisation entre hôtes.
                        \end{column}
                \end{columns}                 
            \end{block}
      \end{column}
      \begin{column}{\sepwid}\end{column}
      \begin{column}{\onecolwid}
          \begin{block}{Résultats}
          	\begin{columns}[t]
          		\begin{column}{\onecolwid}\justify
          			Une implémentation préliminaire, en architecture client-serveur, est offerte en tant qu'extension au logiciel i-score.
          		\end{column}
          	\end{columns}        
            \end{block}
        \end{column}         
   \begin{column}{\sepwid}\end{column}		
\end{columns}

\setbeamerfont*{block title}{size=\Large} 
  \vspace{0.5in}  
  \begin{columns}[t]
    
 \begin{column}{\sepwid}\end{column}
 
 \begin{column}{\onecolwid}
  \begin{beamercolorbox}[wd=\textwidth,colsep=0.05cm]{cboxb}\end{beamercolorbox}
 \end{column}
  
 \begin{column}{\sepwid}\end{column}
  
 \begin{column}{\twocolwid}
  \begin{beamercolorbox}[wd=\textwidth,colsep=0.05cm]{cboxb}\end{beamercolorbox}            
 \end{column}
  
 \begin{column}{\sepwid}\end{column}
 
 \begin{column}{\onecolwid}
  \begin{beamercolorbox}[wd=\textwidth,colsep=0.05cm]{cboxb}\end{beamercolorbox}
 \end{column}         
  
 \begin{column}{\sepwid}\end{column}
\end{columns}
  \vspace{-1ex}  
  \begin{columns}[t]
    \begin{column}{\sepwid}\end{column}
    \begin{column}{\onecolwid}
      \vskip2ex
       \begin{block}{Partitions interactives}
\begin{figure}
    \begin{tikzpicture}[scale=4, every node/.style={scale=0.8}]
\fill (0, 21.6585) circle (0.075) ; % State.1 
\fill (0.801509, 21.6585) circle (0.075) ; % State.2 
\fill (0.801509, 20.5122) circle (0.075) ; % State.3 
\fill (4.1715, 20.5122) circle (0.075) ; % State.4 
\fill (0.801509, 21.387) circle (0.075) ; % State.5 
\fill (3.23104, 21.387) circle (0.075) ; % State.6 
\fill (3.23104, 21.0099) circle (0.075) ; % State.7 
\fill (5, 21.0099) circle (0.075) ; % State.8 
\fill (4.1715, 20.2558) circle (0.075) ; % State.9 
\fill (5, 20.2558) circle (0.075) ; % State.10 
\fill (0.801509, 19.1095) circle (0.075) ; % State.11 
\fill (2.96233, 19.1095) circle (0.075) ; % State.12 
\draw[line width=3pt] (0, 21.6585)  -- (0, 21.6585) ; % TimeNode.0 
\draw[line width=3pt] (0.801509, 21.6585)  -- (0.801509, 19.1095) ; % mule87fens83 
\draw[line width=3pt] (4.1715, 20.5122)  -- (4.1715, 20.2558) ; % zero63aunt55 
\draw[line width=3pt] (3.23104, 21.387)  -- (3.23104, 21.0099) ; % dell37didn59 
\draw[line width=3pt] (5, 21.0099)  -- (5, 20.2558) ; % brow57jill79 
\draw[line width=3pt] (2.96233, 19.1095)  -- (2.96233, 19.1095) ; % many97grid9 
\draw[dashed,line width=3pt] (0, 21.6585)  -- (0.801509, 21.6585) ; % slog45felt83 
\draw[line width=3pt] (0.801509, 20.5122)  -- (4.1715, 20.5122) ; % volume 
\draw[line width=3pt] (0.801509, 20.4122)  -- (4.1715, 20.4122)  -- (4.1715, 19.4122)  -- (0.801509, 19.4122)  -- (0.801509, 20.4122) ;
\draw[line width=3pt] (0.801509, 19.4122)  -- (4.1715, 20.4122) ;
\draw[line width=3pt] (0.801509, 21.387)  -- (3.23104, 21.387) ; % ears55auto57 
\draw[dashed,line width=3pt] (3.23104, 21.0099)  -- (5, 21.0099) ; % cole68beet23 
\draw[line width=3pt] (4.1715, 20.2558)  -- (4.574, 20.2558) ; % nate5just59 
\draw[dashed,line width=3pt] (4.474, 20.2558)  -- (5.71206, 20.2558) ; % nate5just59 
\draw[line width=3pt] (4.714, 20.4088) arc(90:270:0.15) ; % nate5just59 
\draw[line width=3pt] (5.56206, 20.1058) arc(-90:90:0.15) ; % nate5just59 
\draw[line width=3pt] (0.801509, 19.1095)  -- (2.96233, 19.1095) ; % lumiere 
\draw[line width=3pt] (0.801509, 19.0095)  -- (2.96233, 19.0095)  -- (2.96233, 18.0095)  -- (0.801509, 18.0095)  -- (0.801509, 19.0095) ;
\draw[line width=3pt] (0.801509, 18.0095)  -- (2.96233, 19.0095) ;
\draw[line width=3pt] (0.601509, 20.5122)  -- (0.601509, 19.1095) ; % adds31aloe19 
\draw[line width=3pt] (0.601509, 20.5122) arc(180:75:0.2) ; % adds31aloe19 
\draw[line width=3pt] (0.601509, 19.1095) arc(180:285:0.2) ; % adds31aloe19 


\draw (2.1, 21.5199) node {Durée fixe};
\draw (4.104, 21.1099) node {Durée souple};
\draw (2.201509, 20.7) node {Branches conditionnées};
\draw (1.301509, 20) node {Contenu};
\draw (1.301509, 18.6295) node {Contenu};
\draw (5, 19.95) node {Interaction};
\end{tikzpicture}
\caption{Syntaxe d'une partition interactive}
\end{figure}
Un \textbf{langage visuel} de programmation structurée, axé sur l'organisation temporelle de médias.
Des primitives permettent l'écoulement du temps, la synchronisation, le bouclage, et la hiérarchisation des contenus.

Il existe des applications pour la musique interactive, la scénographie, le spectacle vivant, et le contrôle de robots.

Le formalisme propose déjà une répartition bas-niveau par envoi de messages.
On souhaite rajouter une couche d'écriture permettant de la spécifier explicitement dans les partitions.
\end{block}
\vskip1ex
\begin{block}{Applications visées}
    \begin{itemize}
        \item Création de \textbf{murs d'écrans vidéo} synchronisés.
        \item Installations artistiques polyphoniques et ouvertes~: 
        par exemple exploiter la présence de \textbf{plusieurs appareils mobiles} 
        lors de l'écriture.
        \item \textbf{Back-up} à chaud des régies de spectacle.
        \item Réduction du \textit{jitter} sur périphériques embarqués.
    \end{itemize}
\end{block}

\begin{block}{Existant}
\begin{itemize}
    \item \textbf{Modélisation algorithmique} : réseaux de Petri pour synchronisation. 
    \item \textbf{Horloges} : physiques (NTP\cite{mills1991internet}, PTP\cite{peng2009research}), logiques (Lamport\cite{lamport1978time}, vector, matrix) et hybrides\cite{kulkarni2014logical}.
    Approches par intervalles plutôt que par dates : Google TrueTime.
    Horloges adaptées à la gestion du temps musical : Ableton Link, Global Metronome\cite{oda2016global}.
    \item 
    \textbf{Applications musicales réparties} : Ohm Studio, Splice, Supercollider\cite{carot2007networked}.
\end{itemize}
\end{block}
      \vskip2ex
      \begin{block}{Existant}
\textbf{Modélisation algorithmique} : réseaux de Petri pour synchronisation. 
\textbf{Horloges} : NTP, PTP, Hybrid Clocks, Ableton Link.
\textbf{Partage de flux} : NetJACK
\textbf{Applications musicales réparties} : Ohm Studio, Impromptu.

Pas d'outil de répartition des structures musicales.
\end{block}
    \end{column}

    \begin{column}{\sepwid}\end{column}
    \begin{column}{\onecolwid}
        \begin{block}{Implémentation}
Système d'entités adapté pour \textbf{hiérarchie fixée} dans le modèle : tout ne se compose pas avec tout.
Identification unique fortement typée avec cache~: 
\begin{itemize}
    \item Dans un document par chemins : nécessaire pour gestion undo - redo et identification sur réseau.
    Pattern Commande réparti pour \textbf{édition multi-utilisateurs}.
    \item À un niveau de hiérarchie donnée : performance pour itération.
   \end{itemize}

Création de composants fortement ou faiblement typés selon le besoin, et associés à un élément de modèle.

Contrairement à moteurs de jeu, \textbf{pas de synchro des tick rate} car tous les systèmes sont séparés ; certains systèmes peuvent fonctionner sur le même thread ou sur des threads différents. 
Les systèmes peuvent communiquer entre eux.
\end{block}
    \end{column}
    \begin{column}{\sepwid}\end{column}
    \begin{column}{\onecolwid}
        \begin{block}{Synchronisation des micro-structures}
	\begin{itemize}
		\item Mode synchrone
		\item Mode asynchrone
	\end{itemize}
\end{block}
\begin{block}{Expressions et consensus}
\end{block}
     \end{column}
  \begin{column}{\sepwid}\end{column}
  \begin{column}{\onecolwid}
    \vskip2ex
    \begin{block}{Prochains objectifs}
	\begin{itemize}
		\item Évolution vers édition répartie de partitions interactives, pendant leur exécution. Objectif~: permettre à plusieurs régisseurs (son, lumière) de garder la main sur une même exécution.
		\item Comportements de groupe~: utiliser les informations provenant de chaque client, pour générer de nouveaux contenus. Par exemple, étant donné plusieurs téléphones portables, pouvoir utiliser les capteurs intégrés et travailler avec le comportement moyen sur ce groupe.
	\end{itemize}
\end{block}
    \vskip2ex
    \begin{block}{Informations complémentaires}
    	% Site web
    	% Article
      {Articles sur ce sujet~:
      \begin{itemize}
        \item Modèles formels sur lesquels se base i-score~:~\\\cite{allombert_system_2007,arias_modelling_2014}.
        \item Paradigme graphique OSSIA~:~\cite{celerier_ossia:_2015}.
      \end{itemize}
      \vspace{0.1in}\noindent i-score peut être téléchargé librement sur
      \begin{itemize}
        \item \url{www.i-score.org}
      \end{itemize}}
\end{block}
    \vskip2ex
    \begin{block}{Références}
        \footnotesize\printbibliography
    \end{block}
    \vskip2ex
  \end{column}
  \begin{column}{\sepwid}\end{column}			% empty spacer column
 \end{columns}
\end{frame}
\end{document}

